
\section{asd}
\subsection{Background}


The basic principles behind the fuel cell was first demonstrated by Humphry Davy in 1801, and in 1839 William Grove created the first fuel cell. Grove called his invention a gas battery. In 1889, Charles Langer and Ludvig Mond gave the technology the name fuel cell. After this there is a long gap in history of little development on the fuel cell until the 1950s when General Electric created the first Proton-exchange membrane (PEM) fuel cell. Since the 1950s fuel cells have been used to power a lot of different vehicles and other technology. Some exampels of uses are in NASA space missions in 1960s, in submarines by the US Navy in 1980s and in cars by Honda in 2008.

As we are phasing out fossil fuels as energy resource in the transporting sector there have been created a need for other more environmental friendly technologies to take over. In this vacuum fuel cell technology have blossomed. Especially in fields where battery storage is an inefficient way of storing energy.

\subsection{Outline}

In this project we are going to study the PEM fuel cell to get a better understanding of the concept of fuel cells. During our project we are going to design and test a single cell PEM fuel cell. Using Solidworks to create a 3D model of the housing and 3D printers to make the housing.

After printing and assembling the Membrane Electrode Assemblies (MEA), and the housing to a complete single cell fuel cell, we are going to test the voltage and power output of the fuel cell. We are also going to look at the difference in performance of the fuel cell with and without the nickle foam gas diffusion layer.


%As the world faces two big problems, global warming and resource shortage the need for renewable energy sources are increasing. Exactly because of this, the fuel cell has a great appeal as it generates electricity with very low pollution with the fuels consisting of hydrogen and oxygen. This can with optimized technology be a very good alternative to fossil fuels. 