\section{Gas diffusion}

There are many ways of diffusing the gases inside a fuel cell. One way is designing the housing whit etched channels 
to lead the gasses around in the fuel cell. The etched channels are the made to create a long way between the input and output. This way the gas gets distributed across a large part of the membrane before leaving the fuel cell. 

Another way to diffuse the gas is by using a gas diffusing layer. In our PEMFC we used a nickel foam layer to force the gas molecules to spread evenly on the membrane. 

\subsection{Etched channels for gas distribution}

We decided

By creating a distribution pattern in the fuel cell housing unit itself and making holes in the current collectors gas could be distributed to cover the whole area of the electrolyte. To save time during the modeling of the housing parts we decided to use nickel foam as a gas distribution layer, thereby making distribution patterns unnecessary.

\subsection{Flattening nickle foam}

To make the nickel foam fit inside the camper of the PEMFC it had to be flattened. To flatten it we used a hammer and a hard surface. The nickle foam is meant to distribute gas from the supply tap to cover the whole area of the electrolyte. By flattening the nickel foam there is a possibility of decreasing 

By flattening the foam we are decreasing its ability to distribute gas. This could have a effect on the results.  

\subsection{Distribution layer}

Testing shows us that we get a higher voltages while using the nickle foam for gas distribution rather then not. We believe this is because of the carbon paper does not distribute gas enough to cover the whole area of the electrolyte.    

\section{Fuel flow}
\subsection{Cross sectional supply taps}

By creating the two housing units with with "crossed patterned supply taps ??" gas would be supplied at...

\subsection{Gas flow control}

Using a manual supply valve to control the flow of hydrogen with no means to measure a flow of gas in the system there is also no way of accurately controlling that we supplied the same amount of gas flow in each test. By not supplying the same amount of hydrogen in each of the tests there might be either more or less reactions happening, giving different voltage output. There is also no way of telling that the air-pump, which is an aquarium pump, will deliver the same amount of flow at any given time. This might also cause the same concerns as expressed for hydrogen flow.

\section{Moisture management}
\subsection{Electrolyte moisture}
Testing shows us that the fuel cell performs better while the electrolyte is moist over that it does when it is dry. But we have no way of telling if water covers the whole electrolyte when we fill it with water, and if the "level of moisture" is the same in each of the wet tests and each of the dry tests...

\subsection{Damage when drying electrolyte ??}






%   Skriv diskusjonen

%   Spredningen i målinger (ohmic loss)
%   Input/output tapper kunne vært motsatt av hverandre, kunne det vært bedre?
%   Feilkilder som;
%            hvergang
%       Akvariepumpe - nøyaktig?
%       Membran kan bli dårlig av tørking gjentatte ganger?
%       Fuktighet på membranen ved ulike tester